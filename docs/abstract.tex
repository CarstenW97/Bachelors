The goal of this bachelors project was to investigate how the proteome of a system is impacted by thermodynamic driving forces. To that end a coarse grained model was used, simplifying the metabolic network of Echerichia coli.
The system used consisted of the glycolysis reactions as well as the synthesis of glutamine with a few maintenance reactions added to supply all necessary cofactors. 
At the biginning it was planned to look at both the Embden-Meyerhof-Parnas (EMP) and Entner–Doudoroff (ED) but due to time limitations the focus was placed on the EMP pathway.
To observe the changes in this system the ratio of ATP/ADP and NADH/NAD were varied while the glutamine synthesis was used as a stand in for the growth rate of the cell. The glutamine synthesis was chosen, because it is a precursor amino acid for many other amino acids as well as other intracellular metabolites.
This model was then looked at further with concentrations for cofactors ATP, ADP, NADH and NAD being fixed as well as how the model behaves when other intracellular metabolite concentrations are fixed. 
The reuslts of these investigations into how the model bahave under different conditions shows that under almost all of them no changes in the proteome in comparison to the base model can be seen.