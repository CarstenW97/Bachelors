\section{Project model}
The main goal with the model was the investigation of thermodynamics on the proteome. 
This thesis project consists of 30 reaction including the EMP glycolysis, glutamine synthesis, import and export of metabolites as well as as maintanace reactions, shown in \ref{fig:model_map}. 
\begin{figure}[H]
    \centering
    \includegraphics[width=\textwidth]{imgs/model_map.PNG}
    \caption{Schematic of the reaction pathways that make up the model used in this project, showing all reactions, metabolites and cofactors.}
    \label{fig:SActive}
\end{figure}

The flux equations for the model are based on \ref{eq:thermokinetic_flux_forward}, meaning substrate saturation was assumed. This leaves the katalytic constants, enzyme concetration and thermodynamics as the impacting variables.
For reactions the dG values were calculated with the the dG0 for the reactions, the gas constant, temperature and the product and substarte concentrations.
\begin{equation}
    dG == dG0 + RT * (Products - Substrate)
    \label{eq:general_dG}
\end{equation}

With the gas constant, temperature and dG0 being constant values the change in the thermodynamic term comes from changes in substrate and product concentrations. 

It was assumed that most reactions can only go in a forward direction to simplify the model. 
The protome of the system is limited by a density constraint simulating the limited space of a cell. With that no singular enzyme concentration could not exceed a set upper boundary, while also contraining the complete proteome with this limit.
Further the metabolites, fluxes and the dG values were bound, with these boundries being set to be comparable to values found in vivo. 

The base model shows that under a changing ratio of ATP to ADP the proteome stays overall consinstent, with only slight increases and decreases in a few enzymes (see \ref{fig:proteome_upper_atp} and \ref{fig:proteome_lower_atp}). The metabolite concentations change more drametically due to the ratio change, which are shown in \ref{fig:metabolites_upper_atp} and \ref{fig:metabolites_lower_atp}. 
\begin{figure}[H]
    \centering
    \includegraphics[width=\textwidth]{imgs/EMP-model/Upper_Glycolysis_Proteome_ATP_ADP.pdf}
    \caption{Concentrations of the enzymes in the upper glycolysis under the change in ATP/ADP ratio from 0.1 M to 10 M.}
    \label{fig:proteome_upper_atp}
\end{figure}

\begin{figure}[H]
    \centering
    \includegraphics[width=\textwidth]{imgs/EMP-model/Lower_Glycolysis_Proteome_ATP_ADP.pdf}
    \caption{Concentrations of the enzymes in the lower glycolysis under the change in ATP/ADP ratio from 0.1 M to 10 M.}
    \label{fig:proteome_lower_atp}
\end{figure}

\begin{figure}[H]
    \centering
    \includegraphics[width=\textwidth]{imgs/EMP-model/Upper_Glycolysis_Metabolites-ATP_ADP.pdf}
    \cation{Metabolite concentrations of the upper glycolysis over the change of ATP/ADP ratio from 0.1 M to 10 M.}
    \label{fig:metabolites_upper_atp}
\end

\begin{figure}[H]
    \centering
    \includegraphics[width=\textwidth]{imgs/EMP-model/Lower_Glycolysis_Metabolites-ATP_ADP.pdf}
    \cation{Metabolite concentrations of the lower glycolysis over the change of ATP/ADP ratio from 0.1 M to 10 M.}
    \label{fig:metabolites_lower_atp}
\end

 