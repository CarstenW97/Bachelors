\section{Project model}
The main goal with the model was the investigation of thermodynamics on the proteome. 
This thesis project consists of 30 reaction including the EMP glycolysis, glutamine synthesis, import and export of metabolites as well as as maintanace reactions, shown in \ref{fig:model_map}. 
\begin{figure}[H]
    \centering
    \includegraphics[width=0.8\textwidth]{imgs/model_map.PNG}
    \caption{Schematic of the reaction pathways that make up the model used in this project, showing all reactions, metabolites and cofactors.}
    \label{fig:model_map}
\end{figure}

The glucose importer is assumed to act like a PTS, phosphorylating the glucose to glucose-6-phosphate as it is imported into the cell (primary active uniport). The import and export of the other metabolites lactate, ethanol and NH4 are passive transport proteins (secondary active symport).

The flux equations for the model are based on \ref{eq:thermokinetic_flux_forward}, meaning substrate saturation was assumed. This leaves the katalytic constants, enzyme concetration and thermodynamics as the impacting variables.
For reactions the dG values were calculated with the the dG0 for the reactions, the gas constant, temperature and the product and substarte concentrations.
\begin{equation}
    dG == dG0 + RT * (Products - Substrate)
    \label{eq:general_dG}
\end{equation}

With the gas constant, temperature and dG0 being constant values the change in the thermodynamic term comes from changes in substrate and product concentrations. 

The objective to maximize in the model was the glutamine synthesis as a stand in for the growth rate of the cell. Glutamine was chosen because it is a precursor to many other amino acids as well as a substrate for other metabolic reactions that are important for cell growth.
It was assumed that most reactions can only go in a forward direction to simplify the model. 
The protome of the system is limited by a density constraint simulating the limited space of a cell. With that no singular enzyme concentration could not exceed a set upper boundary, while also contraining the complete proteome with this limit.
For the baseline model no intracellular concentrations, fluxes or dG of reaction were fixed to a specific value. The metabolite concentrations are fixed between log(1e-9) and log(100e-3) (log(M)), the fluxes between -1 mmol/gDW/h and 100 mmol/gDW/h (gDW = gramm dry weight) and dG of reaction between -80 kJ/mol and 1 kJ/mol. 

\section{Results}
The base model shows that under a changing ratio of ATP to ADP the proteome stays overall consinstent, with only slight increases and decreases in a few enzymes (see \ref{fig:proteome_upper_atp} and \ref{fig:proteome_lower_atp}). The metabolite concentations change more drametically due to the ratio change, which are shown in \ref{fig:metabolites_upper_atp} and \ref{fig:metabolites_lower_atp}. 
\begin{figure}[H]
    \centering
    \includegraphics[width=0.8\textwidth]{imgs/EMP-model/Upper_Glycolysis_Proteome_ATP_ADP.pdf}
    \caption{Concentrations of the enzymes in the upper glycolysis under the change in ATP/ADP ratio from 0.1 M to 10.0 M.}
    \label{fig:proteome_upper_atp}
\end{figure}

\begin{figure}[H]
    \centering
    \includegraphics[width=0.8\textwidth]{imgs/EMP-model/Lower_Glycolysis_Proteome_ATP_ADP.pdf}
    \caption{Concentrations of the enzymes in the lower glycolysis under the change in ATP/ADP ratio from 0.1 M to 10.0 M.}
    \label{fig:proteome_lower_atp}
\end{figure}

\begin{figure}[H]
    \centering
    \includegraphics[width=0.8\textwidth]{imgs/EMP-model/Upper_Glycolysis_Metabolites_ATP_ADP.pdf}
    \caption{Metabolite concentrations of the upper glycolysis over the change of ATP/ADP ratio from 0.1 M to 10.0 M.}
    \label{fig:metabolites_upper_atp}
\end{figure}

\begin{figure}[H]
    \centering
    \includegraphics[width=0.8\textwidth]{imgs/EMP-model/Lower_Glycolysis_Metabolites_ATP_ADP.pdf}
    \caption{Metabolite concentrations of the lower glycolysis over the change of ATP/ADP ratio from 0.1 M to 10.0 M.}
    \label{fig:metabolites_lower_atp}
\end{figure}

Similarly the metabolite concentrations for the NADH/NAD ratio change in a comparabble way to the ones seen for the ATP/ADP ratio. The change seen for the upper glycolysis (\ref{fig:metabolites_upper_nadh}) is less rapid and the concentrations of g6p and f6p do not drop as low as seen in \ref{fig:metabolites_upper_atp}. For the lower glycolysis the metabolite concentration change (\ref{fig:metabolites_lower_nadh}) is smoother for the NADH/NAD ratio when compared to that of the ATP/ADP ratio (\ref{fig:metabolites_lower_atp}), but the genreal behavior is similar.
\begin{figure}[H]
    \centering
    \includegraphics[width=0.8\textwidth]{imgs/EMP-model/Upper_Glycolysis_Metabolites_NADH_NAD.pdf}
    \caption{Metabolite concentrations of the upper glycolysis over the change of NADH/NAD ratio from 0.05 M to 10.0 M.}
    \label{fig:metabolites_upper_nadh}
\end{figure}

\begin{figure}[H]
    \centering
    \includegraphics[width=0.8\textwidth]{imgs/EMP-model/Lower_Glycolysis_Metabolites_NADH_NAD.pdf}
    \caption{Metabolite concentrations of the lower glycolysis over the change of NADH/NAD ratio from 0.05 M to 10.0 M.}
    \label{fig:metabolites_lower_nadh}
\end{figure}

As it was seen for the proteome of the ATP/ADP ratio the proteome stays relatively constant over all different values for the NADH/NAD ratio. The relative concentrations are for both the upper and lower glycolysis are even comparable to the ones seen in \ref{fig:proteome_upper_atp} and \ref{fig:proteome_lower_atp}.
\begin{figure}[H]
    \centering
    \includegraphics[width=0.8\textwidth]{imgs/EMP-model/Upper_Glycolysis_Proteome_NADH_NAD.pdf}
    \caption{Concentrations of the enzymes in the upper glycolysis under the change in ATP/ADP ratio from 0.1 M to 10.0 M.}
    \label{fig:proteome_upper_nadh}
\end{figure}

\begin{figure}[H]
    \centering
    \includegraphics[width=0.8\textwidth]{imgs/EMP-model/Lower_Glycolysis_Proteome_NADH_NAD.pdf}
    \caption{Concentrations of the enzymes in the lower glycolysis under the change in ATP/ADP ratio from 0.1 M to 10.0 M.}
    \label{fig:proteome_lower_nadh}
\end{figure}

The relatively constant proteome can be explained by the big changes in metabolite concentrations. With these concentration having a direct effect on the termodynamic term of the flux calculation, the fluxes still change but only due to the thermodynamic term. Thus the fluxes will be adjusted to the new conditions without the need for the system to produce or destroy any enzymes. In a cell this would also make sense, due to the fact that the synthesis of enzymes is expensive, while using or importing/exporting more metabolites to change the fluxes of reactions has a similar effect with less cost to the cell.
This conection can be seen in \ref{fig:upper_fluxes_atp} which shows that the fluxes for the ractions stay the same even under this change in ATP/ADP ratio.

\begin{figure}[H]
    \centering
    \includegraphics[width=0.8\textwidth]{imgs/EMP-model/Upper_Glycolysis_Fluxes_ATP_ADP.pdf}
    \caption{Plot of the change in fluxes of the upper glycolysis reactions due to the change in ATP/ADP ratio.}
    \label{fig:upper_fluxes_atp}
\end{figure}

This effect can not only be seen when looking at a change in the ATP/ADP ratio but also for a change in the ratio of NADH/NAD (\ref{fig:upper_fluxs_nadh}).
\begin{figure}[H]
    \centering
    \includegraphics[width=0.8\textwidth]{imgs/EMP-model/Upper_Glycolysis_Fluxes_NADH_NAD.pdf}
    \caption{Plot of the change in fluxes of the upper glycolysis reactions due to the change in NADH/NAD ratio from 0.05 M to 10.0 M.}
    \label{fig:upper_fluxes_nadh}
\end{figure}

With the fuxes showing a stable relative distribution, the relative proteome showing little to no fluctuations th only part of the flux equation that can be changed is the termodynamic part. This term should change due to the differences in metabolite concentrations which do factor into the dG of reaction. 
When we look at the change in dG of reaction for the upper glycolysis under both ATP/ADP (\ref{fig:upper_dG_atp}) and NADH/NAD (\ref{fig:upper_dG_nadh}) ration variation we can see that even these dG of reactions show little to no change. These small changes could be attributed to the fact that the metabolite concentration changes are too small to make a bigger difference. This can be seen in the fact that the relative change in dG of reaction is much bigger when the metaboite concentrations vary more strongly (see \ref{fig:upper_dG_pep_fixed_atp} and \ref{fig:upper_dG_pep_fixed_nadh}).
\begin{figure}[H]
    \centering
    \includegraphics[width=0.8\textwidth]{imgs/EMP-model/Upper_Glycolysis_dG_ATP_ADP.pdf}
    \caption{Plot of dG of reaction of the Upper glycolysis reactions due to the change in ATP/ADP ratio from 0.1 M to 10.0 M.}
    \label{fig:upper_dG_atp}
\end{figure}

\begin{figure}[H]
    \centering
    \includegraphics[width=0.8\textwidth]{imgs/EMP-model/Upper_Glycolysis_dG_NADH_NAD.pdf}
    \caption{Plot of dG of reaction of the Upper glycolysis reactions due to the change in NADH/NAD ratio from 0.05 M to 10.0 M.}
    \label{fig:upper_dG_nadh}
\end{figure}

The dG of reaction shows a stonger variation in it's relative distribution when looking at the model under a varying NADH/NAD ratio. 

These behaiviors are not only observable for the fluxes and concentrations of the glycolysis reactions. They can also be seen if we compare the proteome and fluxes between the ratios for the glutamine synthesis, which means this behaivior is not exclusive to the gycolosis but fundamental to the system.
In some of these plots not all concentrations are visible, but this is explainable with the fact that those concentrations of enzymes or metabolites or the fluxes are so low that they are comperativly small and don't show up in a graph that only shows relative concentrations.

The trend of a mostly stable relative proteome continues even when the cofactors are fixed, while the matabolite concentrations show a similar behavior to the system without fixed cofactor concentrations. This also has the same effect on the relative fuxes, meaning they do not change as a result. The only thing that shows a difference to the base model are the relative cofactor concentrations themselves, which makes sense, if they are more limited in their range than before that would mean that the ratio between them change. They also show similar changes in cofactor concentrations between themselves, with the plots for fixed ATP (\ref{fig:cofactor_atp_fixed_atp}) and ADP (\ref{fig:cofactor_adp_fixed_atp}) being almost the same, with the slight difference in the NADH concentration maximum. 
The values to which the cofactors are fixed were taken from Bennett, et. al. 2009 \cite{bennett2009absolute}. With these fixed ATP and ADP concentrations the ratio change for NADH/NAD also show generally the same trend, with the ATP concentration being lower when the ATP concentration is fixed. The relative NADH and NAD concentrations resembeling those of the system with fixed ATP and ADP.
\begin{figure}[H]
    \centering
    \includegraphics[width=0.8\textwidth]{imgs/EMP-model/ATP_fixed/Cofactor_Metabolites_ATP_ADP.pdf}
    \caption{Change of the Cofactor concentration with a change in ATP/ADP ratio (0.1 M to 10.0 M.) and ADP fixed between 1/4.0 * 9.6e-3 M and 4.0 * 9.6e-3 M.}
    \label{fig:cofactor_atp_fixed_atp}
\end{figure}

\begin{figure}[H]
    \centering
    \includegraphics[width=0.8\textwidth]{imgs/EMP-model/ADP_fixed/Cofactor_Metabolites_ATP_ADP.pdf}
    \caption{Change of the Cofactor concentration with a change in ATP/ADP ratio (0.1 M to 10.0 M.) and ADP fixed to 4.0 * 5.6e-4 M.}
    \label{fig:cofactor_adp_fixed_atp}
\end{figure}

Comperable to the the system with fixed ATP and ADP, similar trends are seen in when NADH or NAD are fixed. For this NADH was fixed to 4.0 * 8.3e-5 M and the NAD concentration between 1/4.0 * 2.6e-3 M and 4.0 * 2.6e-3 M. Here the general trends are similar, but unlike in the system with fixed ATP and ADP the relative concentration differ between the fixed NADH and NAD. Specifically the relatvie ATP and ADP concentration are distictly higher when the NADH concentration is fixed. This can be explained by the fact that the fixed NADH concentation is about 1e-2 M lower than the fixed NAD concentartion.

Further the effect of fixing the internal metabolite concentrations under the change in ATP/ADP ratio was investegated, as well as the NADH/NAD ratio. As with the base model and the fixed cofactor models, fixing metabolites to specific values shows little effect on the proteome and fluxes. The metabolite and cofactor concentrations again show combarable trends between systems. Looking at the examle of setting fructose-1,6-bishosphate to a fixed range of 1/4.0 * 1.5e-2 M and 4.0 * 1.5e-2 M, we find that the relative proteome fractions change in the same way as the base model. Shown are examples for this behavior for the upper glycolysis enzymes in \ref{fig:upper_proteome_fdp_fixed_atp} and \ref{fig:upper_proteome_fdp_fixed_nadh}.
\begin{figure}[H]
    \centering
    \includegraphics[width=0.8\textwidth]{imgs/EMP-model/FDP_fixed/Upper_Glycolysis_Proteome_ATP_ADP.pdf}
    \caption{Relative proteome concentration of the upper glycolysis enzymes under ATP/ADP ratio change and fixed fructose-1,6-bisphosphate.}
    \label{fig:upper_proteome_fdp_fixed_atp}
\end{figure}

\begin{figure}[H]
    \centering
    \includegraphics[width=0.8\textwidth]{imgs/EMP-model/FDP_fixed/Upper_Glycolysis_Proteome_NADH_NAD.pdf}
    \caption{Relative proteome concentration of the upper glycolysis enzymes under NADH/NAD ratio change and fixed fructose-1,6-bisphosphate.}
    \label{fig:upper_proteome_fdp_fixed_nadh}
\end{figure}

Fixing other metabolite concentrations for example that of phosphoenolpyruvate leads to very similar results in relative proteome concentrations. Complementary the metabolite concentartion changes are mostly in line with the chages seen in the base model when the ATP/ADP and NADH/NAD ratio are varied. These relative concentration changes do show differences between the different fixed metabolite concentrations. The general trends are comparablewhen looking at the ATP/ADP ratio (\ref{fig:upper_metabolites_fdp_fixed_atp}), but for a change in NADH/NAD ratio leads to bigger differences (\ref{fig:upper_metabolites_fdp_fixed_nadh}). 
\begin{figure}[H]
    \centering
    \includegraphics[width=0.8\textwidth]{imgs/EMP-model/FDP_fixed/Upper_Glycolysis_Metabolites_ATP_ADP.pdf}
    \caption{Relative metabolite concentration in the upper glycolysis under ATP/ADP ratio change and fixed fructose-1,6-bisphosphate.}
    \label{fig:upper_metabolites_fdp_fixed_atp}
\end{figure}

\begin{figure}[H]
    \centering
    \includegraphics[width=0.8\textwidth]{imgs/EMP-model/FDP_fixed/Upper_Glycolysis_Metabolites_NADH_NAD.pdf}
    \caption{Relative metabolite concentration in the upper glycolysis under NADH/NAD ratio change and fixed fructose-1,6-bisphosphate.}
    \label{fig:upper_metabolites_fdp_fixed_nadh}
\end{figure}

These models with fixed metabolite concentration also show similar changes to the dG of reaction as the base model, which due to their general similarity to the behavior shown by the base model is not surprising.

Bigger changes can be obsereved with phosphoenolpyruvate (PEP) being fixed to a range between 1/4.0 * 1.8e-4 M and 4.0 * 1.8e-4 M. Here the relative proteome as well as the relative metabolite concentrations show changes. With the most promenent changes happening to the relative proteome of the upper (\ref{fig:upper_proteome_pep_fixed_atp}) and lower glycolysis (\ref{fig:lower_proteome_pep_fixed_atp}) as well as the gultamine synthesis (\ref{fig:glutamine_proteome_pep_fixed_atp}) under ATP/ADP ratio changes as well as the realative metabolite concentrations to the lower glycolysis metabolites under both ratios (\ref{fig:lower_metabolites_pep_fixed_atp}, \ref{fig:lower_metabolites_pep_fixed_nadh}) and the upper glycolysis with NADH/NAD ratio change (\ref{fig:upper_metabolite_pep_fixed_nadh}). 
\begin{figure}[H]
    \centering
    \includegraphics[width=0.8\textwidth]{imgs/EMP-model/PEP_fixed/Upper_Glycolysis_Proteome_ATP_ADP.pdf}
    \caption{Relative proteome of the upper glycolysis with a fixed PEP concentration and a variying ATP/ADP ratio.}
    \label{fig:upper_proteome_pep_fixed_atp}
\end{figure}

\begin{figure}[H]
    \centering
    \includegraphics[width=0.8\textwidth]{imgs/EMP-model/PEP_fixed/Glutamine_synthesis_Proteome_ATP_ADP.pdf}
    \caption{Relative proteome for the glutamine synthesis with a fixed PEP concentration and a variying ATP/ADP ratio.}
    \label{fig:glutamine_proteome_pep_fixed_atp}
\end{figure}

\begin{figure}[H]
    \centering
    \includegraphics[width=0.8\textwidth]{imgs/EMP-model/PEP_fixed/Lower_Glycolysis_Proteome_ATP_ADP.pdf}
    \caption{Relative proteome of the lower glycolysis with a fixed PEP concentration and a variying ATP/ADP ratio.}
    \label{fig:lower_proteome_pep_fixed_atp}
\end{figure}

\begin{figure}[H]
    \centering
    \includegraphics[width=0.8\textwidth]{imgs/EMP-model/PEP_fixed/Lower_Glycolysis_Metabolites_ATP_ADP.pdf}
    \caption{Relative metabolite concentrations of the lower glycolysis with a fixed PEP concentration and a variying ATP/ADPratio.}
    \label{fig:lower_metabolites_pep_fixed_atp}
\end{figure}

\begin{figure}[H]
    \centering
    \includegraphics[width=0.8\textwidth]{imgs/EMP-model/PEP_fixed/lower_Glycolysis_Metabolites_NADH_NAD.pdf}
    \caption{Relative metabolite concentrations of the lower glycolysis with a fixed PEP concentration and a variying NADH/NAD ratio.}
    \label{fig:lower_metabolites_pep_fixed_nadh}
\end{figure}

\begin{figure}[H]
    \centering
    \includegraphics[width=0.8\textwidth]{imgs/EMP-model/PEP_fixed/Upper_Glycolysis_Metabolites_NADH_NAD.pdf}
    \caption{Relative metabolite concentrations of the upper glycolysis with a fixed PEP concentration and a variying NADH/NAD ratio.}
    \label{fig:upper_metabolite_pep_fixed_nadh}
\end{figure}

These models show the same relative flux distributions as the base model, and those with fixed cofactor concentrations. This supports the assumption that the fluxes are kept stable due to the change in metabolite concentrations, while the relative enzyme concentrations stay the same.
For this model with fixed PEP concentration we can see that the dG of reaction varies stonger than in the base model.
\begin{figure}[H]
    \centering
    \includegraphics[width=0.8\textwidth]{imgs/EMP-model/PEP_fixed/Upper_Glycolysis_dG_ATP_ADP.pdf}
    \caption{Relative dG of reaction of the upper glycolysis with a fixed PEP concentration and a variying ATP/ADP ratio.}
    \label{fig:upper_dG_pep_fixed_atp}
\end{figure}

\begin{figure}[H]
    \centering
    \includegraphics[width=0.8\textwidth]{imgs/EMP-model/PEP_fixed/Upper_Glycolysis_dG_NADH_NAD.pdf}
    \caption{Relative dG of reaction of the upper glycolysis with a fixed PEP concentration and a variying NADH/NAD ratio.}
    \label{fig:upper_dG_pep_fixed_nadh}
\end{figure}

\section{Discussion}
As the results have shown, the base models realative proteome is stable over the varying ratios of ATP/ADP and NADH/NAD, as are the relative fluxes of the reactions of the system. The results also showed that the observable concentations for both entzymes and metabolites as well as the fluxes of the base model are mostly adhered to when the internal concentrations of metabolites and cofactors are are fixed. 
The stable fluxes can be attrubuted to the fact that the metabolic concentrations are used as a buffer to keep the fluxes constant. Meaning as a result of the changes in the system - namely the change in ATP/ADP and NADH/NAD ratios - the reactions use different amounts of metabolites to keeep the fluxes stable. With these concentrations having an effect on the thermodynamic part of the flux equation it changes to accomadate the new conditions of the system. Due to the fact that the thermodynamic term already adjusts the flux equation to be stable there is no need for  a change in enzyme concentration.
This behaivior can be explained by assuming that it is less cost efficient for the cell to synthesize or desythesize enzymes and it is better to use up different amounts of the metabolites. 

There is one metabolite PEP which lead to different results when it's concentration was. These differences can be explained with the fact that PEP is the first metabolite after the reaction sof the glycolysis that act as thermodynamic bottlenecks. It is also used for multiple reactions with both products (pyruvate and oxaloacetate) being important substrates for the glutamine synthesis. The combination of these factors leads to an increase in importance of PEP for the system.
The extreme incrase in relative glucose-6-phosphate concentration under a fixed PEP value for the varying NADH/NAD ratio could be attributed to the low amount of NADH and NAD in the system slowing down the reactions following GLCpts.  
The only other concentrations that change throughout these model are those of the cofactors, which can be mostly attributed to the fact that the varying ratio between the two cofactor pairs was the main methid uunder which the model was evaluated.
 
The next step after this work could be to see how a model based on the ED glycolysis pathway behaves under the conditions this model was exposed to. If the assumption that the ED pathway is less thermodynamically constrained due to the reduction of thermodynamic botllenecks lower enzyme concentrations should be observable when both pathways are compared at the same reaction fluxes.
Further this model could be extended to be a self replicatior model to see if and how this would affect the model, specifically the proteome which would be the main component impacted by this addition. For this a similar setup as the one described above in could be used. With this glutamine would serve a similar role to it's in vivo counterpart as an important precursor for many other cell metabolites.
