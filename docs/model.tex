\section{Project model}
The main goal with the model was the investigation of thermodynamics on the proteome. 
This thesis project consists of 30 reaction including the EMP glycolysis, glutamine synthesis, import and export of metabolites as well as as maintanace reactions, shown in \ref{fig:model_map}. 
\begin{figure}[H]
    \centering
    \includegraphics[width=\textwidth]{imgs/model_map.PNG}
    \caption{Schematic of the reaction pathways that make up the model used in this project, showing all reactions, metabolites and cofactors.}
    \label{fig:model_map}
\end{figure}

The flux equations for the model are based on \ref{eq:thermokinetic_flux_forward}, meaning substrate saturation was assumed. This leaves the katalytic constants, enzyme concetration and thermodynamics as the impacting variables.
For reactions the dG values were calculated with the the dG0 for the reactions, the gas constant, temperature and the product and substarte concentrations.
\begin{equation}
    dG == dG0 + RT * (Products - Substrate)
    \label{eq:general_dG}
\end{equation}

With the gas constant, temperature and dG0 being constant values the change in the thermodynamic term comes from changes in substrate and product concentrations. 

The objective to maximize in the model was the glutamine synthesis as a stand in for the growth rate of the cell. Glutamine was chosen because it is a precursor to many other amino acids as well as a substrate for other metabolic reactions that are important for cell growth.
It was assumed that most reactions can only go in a forward direction to simplify the model. 
The protome of the system is limited by a density constraint simulating the limited space of a cell. With that no singular enzyme concentration could not exceed a set upper boundary, while also contraining the complete proteome with this limit.
For the baseline model no intracellular concentrations, fluxes or dG of reaction were fixed to a specific value. 

The base model shows that under a changing ratio of ATP to ADP the proteome stays overall consinstent, with only slight increases and decreases in a few enzymes (see \ref{fig:proteome_upper_atp} and \ref{fig:proteome_lower_atp}). The metabolite concentations change more drametically due to the ratio change, which are shown in \ref{fig:metabolites_upper_atp} and \ref{fig:metabolites_lower_atp}. 
\begin{figure}[H]
    \centering
    \includegraphics[width=\textwidth]{imgs/EMP-model/Upper_Glycolysis_Proteome_ATP_ADP.pdf}
    \caption{Concentrations of the enzymes in the upper glycolysis under the change in ATP/ADP ratio from 0.1 M to 10.0 M.}
    \label{fig:proteome_upper_atp}
\end{figure}

\begin{figure}[H]
    \centering
    \includegraphics[width=\textwidth]{imgs/EMP-model/Lower_Glycolysis_Proteome_ATP_ADP.pdf}
    \caption{Concentrations of the enzymes in the lower glycolysis under the change in ATP/ADP ratio from 0.1 M to 10.0 M.}
    \label{fig:proteome_lower_atp}
\end{figure}

\begin{figure}[H]
    \centering
    \includegraphics[width=\textwidth]{imgs/EMP-model/Upper_Glycolysis_Metabolites_ATP_ADP.pdf}
    \cation{Metabolite concentrations of the upper glycolysis over the change of ATP/ADP ratio from 0.1 M to 10.0 M.}
    \label{fig:metabolites_upper_atp}
\end

\begin{figure}[H]
    \centering
    \includegraphics[width=\textwidth]{imgs/EMP-model/Lower_Glycolysis_Metabolites_ATP_ADP.pdf}
    \cation{Metabolite concentrations of the lower glycolysis over the change of ATP/ADP ratio from 0.1 M to 10.0 M.}
    \label{fig:metabolites_lower_atp}
\end

Similarly the metabolite concentrations for the NADH/NAD ratio change in a comparabble way to the ones seen for the ATP/ADP ratio. The change seen for the upper glycolysis (\ref{fig:metabolites_upper_nadh}) is less rapid and the concentrations of g6p and f6p do not drop as low as seen in \ref{metabolites_upper_atp}. For the lower glycolysis the metabolite concentration change (\ref{fig:metabolites_lower_nadh}) is smoother for the NADH/NAD ratio when compared to that of the ATP/ADP ratio (\ref{fig:metabolites_lower_atp}), but the genreal behavior is similar.
\begin{figure}[H]
    \centering
    \includegraphics[width=\textwidth]{imgs/EMP-model/Upper_Glycolysis_Metabolites_NADH_NAD.pdf}
    \cation{Metabolite concentrations of the upper glycolysis over the change of NADH/NAD ratio from 0.05 M to 10.0 M.}
    \label{fig:metabolites_upper_nadh}
\end

\begin{figure}[H]
    \centering
    \includegraphics[width=\textwidth]{imgs/EMP-model/Lower_Glycolysis_Metabolites_NADH_NAD.pdf}
    \cation{Metabolite concentrations of the lower glycolysis over the change of NADH/NAD ratio from 0.05 M to 10.0 M.}
    \label{fig:metabolites_lower_nadh}
\end 

AS it was seen for the proteome of the ATP/ADP ratio the proteome stays relatively constant over all different values for the NADH/NAD ratio. The relative concentrations are for both the upper and lower glycolysis are even comparable to the ones seen in \ref{fig:proteome_upper_atp} and \ref{fig:proteome_lower_atp}.
\begin{figure}[H]
    \centering
    \includegraphics[width=\textwidth]{imgs/EMP-model/Upper_Glycolysis_Proteome_NADH_NAD.pdf}
    \caption{Concentrations of the enzymes in the upper glycolysis under the change in ATP/ADP ratio from 0.1 M to 10.0 M.}
    \label{fig:proteome_upper_nadh}
\end{figure}

\begin{figure}[H]
    \centering
    \includegraphics[width=\textwidth]{imgs/EMP-model/Lower_Glycolysis_Proteome_NADH_NAD.pdf}
    \caption{Concentrations of the enzymes in the lower glycolysis under the change in ATP/ADP ratio from 0.1 M to 10.0 M.}
    \label{fig:proteome_lower_nadh}
\end{figure}

The relatively constant proteome can be explained by the big changes in metabolite concentrations. With these concentration having a direct effect on the termodynamic term of the flux calculation, the fluxes still change but only due to the thermodynamic term. Thus the fluxes will be adjusted to the new conditions without the need for the system to produce or destroy any enzymes. In a cell this would also make sense, due to the fact that the synthesis of enzymes is expensive, while using or importing/exporting more metabolites to change the fluxes of reactions has a similar effect with less cost to the cell.
This conection can be seen in \ref{fig:upper_fluxes_atp} which shows that the fluxes for the ractions stay the same even under this change in ATP/ADP ratio.

\begin{figure}[H]
    \centering
    \includegraphics[width=\textwidth]{imgs/EMP-model/Upper_Glycolysis_Fluxes_ATP_ADP.pdf}
    \caption{Plot of the change in fluxes of the upper glycolysis reactions due to the change in ATP/ADP ratio.}
    \label{fig:upper_fluxes_atp}
\end{figure}

This effect can not only be seen when looking at a change in the ATP/ADP ratio but also for a change in the ratio of NADH/NAD (\ref{fig:upper_fluxs_nadh}).
\begin{figure}[H]
    \centering
    \includegraphics[width=\textwidth]{imgs/EMP-model/Upper_Glycolysis_Fluxes_ATP_ADP.pdf}
    \caption{Plot of the change in fluxes of the upper glycolysis reactions due to the change in NADH/NAD ratio from 0.05 M to 10.0 M.}
    \label{fig:upper_fluxes_nadh}
\end{figure}

These behaiviors are not only observable for the fluxes and concentrations of the glycolysis reactions. They can also be seen if we compare the proteome and fluxes between the ratios for the glutamine synthesis, which means this behaivior is not exclusive to the gycolosis but fundamental to the system.
In some of these plots not all concentrations are visible, but this is explainable with the fact that those concentrations of enzymes or metabolites or the fluxes are so low that they are comperativly small and don't show up in a graph that only shows relative concentrations.

The trend of a mostly stable relative proteome continues even when the cofactors are fixed, while the matabolite concentrations show a similar behavior to the system without fixed cofactor concentrations. This also has the same effect on the relative fuxes, meaning they do not change as a result. The only thing that shows a difference to the base model are the relative cofactor concentrations themselves, which makes sense, if they are more limited in their range than before that would mean that the ratio between them change. They also show similar changes in cofactor concentrations between themselves, with the plots for fixed ATP (\ref{fig:cofactor_atp_fixed_atp}) and ADP (\ref{fig:cofactor_adp_fixed_atp}) being almost the same, with the slight difference in the NADH concentration maximum. 
The values to which the cofactors are fixed were taken from \cite{bennett2009absolute}. With these fixed ATP and ADP concentrations the ratio change for NADH/NAD also show generally the same trend, with the ATP concentration being lower when the ATP concentration is fixed. The relative NADH and NAD concentrations resembeling those of the system with fixed ATP and ADP.
\begin{figure}[H]
    \centering
    \includegraphics[width=\textwidth]{imgs/EMP-model/ATP_fixed/Cofactor_Metabolites_ATP_ADP.pdf}
    \caption{Change of the Cofactor concentration with a change in ATP/ADP ratio (0.1 M to 10.0 M.) and ADP fixed between 1/4.0 * 9.6e-3 M and 4.0 * 9.6e-3 M.}
    \label{fig:cofactor_atp_fixed_atp}
\end{figure}

\begin{figure}[H]
    \centering
    \includegraphics[width=\textwidth]{imgs/EMP-model/ADP_fixed/Cofactor_Metabolites_ATP_ADP.pdf}
    \caption{Change of the Cofactor concentration with a change in ATP/ADP ratio (0.1 M to 10.0 M.) and ADP fixed to 4.0 * 5.6e-4 M.}
    \label{fig:cofactor_adp_fixed_atp}
\end{figure}

Comperable to the the system with fixed ATP and ADP, similar trends are seen in when NADH or NAD are fixed. For this NADH was fixed to 4.0 * 8.3e-5 M and the NAD concentration between 1/4.0 * 2.6e-3 M and 4.0 * 2.6e-3 M. Here the general trend is similar, but 

% Both these ratio changes show different behavior when the system is subjected to a change in extracellular concentrations, with the ATP/ADP ration being mostly uneffected by all of these concenrations. The NADH/NAD ratio does show quite drastic changes when for example the acetate concentration is changed, shown in \ref{fig:ratio_change_ac}. This behavior to the change in acetate can be explained with the fact that an increased extracellular acetate concentration leads to a higher rate of import. Due to the fact that the acetate is turned into Acetyl-CoA other metabolites that are turned into Acetyl-CoA under the use of NAD are needed less. 
% Subsequently an increase for a substrate that needs NAD to be turned into Acetyl-CoA leads to an increase in the NADH/NAD ratio, which can be seen when increasing the extracallular concentration of ethanol form 1e-5 mol to 1e-7 mol (\ref{fig:ratio_change_ethanol}).
% \begin{figure}[H]
%     \centering
%     \includegraphics[width=\textwidth]{imgs/EMP-model/Concentration_changes/Ratio_change_Acetate.pdf}
%     \cation{Change of the ATP/ADP and NADH/NAD ratios when the acetate concentration is increased from 1e-17 mol to 1e-15 mol.}
%     \label{fig:ratio_change_ac}
% \end
% 
% \begin{figure}[H]
%     \centering
%     \includegraphics[width=\textwidth]{imgs/EMP-model/Concentration_changes/Ratio_change_Ethanol.pdf}
%     \cation{Change of the ATP/ADP and NADH/NAD ratios when the ethanol concentration is changed.}
%     \label{fig:ratio_change_ethanol}
% \end
% 
% When looking at the increase in an extracellular concentration that is not directly conneted to reactions that need one of the cofactors NADH, NAD, ATP or ADP we can see that their ratios are mostly unaffected by this concentration change. This can be seen on the example of NH4 which is only relevant in the glutamate and glutamine synthesis reactions (\ref{fig:model_map}).
% \begin{figure}[H]
%     \centering
%     \includegraphics[width=\textwidth]{imgs/EMP-model/Concentration_changes/Ratio_change_NH4.pdf}
%     \cation{Change of the ATP/ADP and NADH/NAD ratios when the NH4 concentration is changed in a range from 0.001 mol to 0.1 mol.}
%     \label{fig:ratio_change_nh4}
% \end
% The small decerease in both their ratios can be explained with the fact that with a higher concentration of NH4 the glutamine reactions is less restricted by NH4. That also means the reaction can happen faster if the other metabolites are suffeciently supplied. Due to the fact that the glutamine synthesis is the objective to maximize in the model this will lead to a higher rate of through put for the proceeding reactions. With this a higher amount of ATP is necessary increasing the ADP concentration and lowering the ATP/ADP ration. Similarly more NAD is needed, which means the maintenance reaction that turns NADH into NAD happens more frequently lowering the NADH concentration and thus the ratio.
% 
% To investigate the effect of the termodynamic term on the system we can also look at the effect that a fixed dG has. If only the ATP/ADP ratio is varied the dG value of the FBA reaction was around -3.0 kJ/mol. Fixing this value as -10 kJ/mol has little impact on the proteome and metabolite concentrations when varying the ATP/ADP ratio as before. It does have an impact on the concentration of the cofactors, with ATP and ADP concentrations being near zero (\ref{fig:cofactor_fba_atp}). In comparison the base model shows a gradual shift in ATP and ADP concentrations, with ADP derasing and ATP increasing with the increase in their ratio (\ref{fig:cofactor_atp}).
% \begin{figure}[H]
%     \centering
%     \includegraphics[width=\textwidth]{imgs/EMP-model/dG_FBA/Cofactor_Metabolites_ATP_ADP.pdf}
%     \cation{Cofactor concentration change under fixed dG for the FBA reaction at -10 kJ/mol and a change in ATP/ADP ratio from 0.1 M to 10 M.}
%     \label{fig:cofactor_fba_atp}
% \end
% 
% \begin{figure}[H]
%     \centering
%     \includegraphics[width=\textwidth]{imgs/EMP-model/Cofactor_Metabolites_ATP_ADP.pdf}
%     \cation{Cofactor concentration change under a change in ATP/ADP ratio from 0.1 M to 10 M.}
%     \label{fig:cofactor_atp}
% \end
% 
% This difference in concenrations can be explained with the fact that FBA is one of the bottelneck reactions in the EMP glycolysis. When this bottleneck is reduced by this increase in dG the following reactions happen at a greater rate meaning more ATP and ADP are required.
% When varying the NADH/NAD ratio with this fixed dG value for the FBA reaction the difference to the model without a fixed value is less pronounced. The overall trend is the same, but the change in concentration is higher.
 