\section{Project model}
The main goal with the model was the investigation of thermodynamics on the proteome. 
This thesis project consists of 30 reaction including the EMP glycolysis, glutamine synthesis, import and export of metabolites as well as as maintanace reactions, shown in \ref{fig:model_map}. 
\begin{figure}[H]
    \centering
    \includegraphics[width=\textwidth]{imgs/model_map.PNG}
    \caption{Schematic of the reaction pathways that make up the model used in this project, showing all reactions, metabolites and cofactors.}
    \label{fig:model_map}
\end{figure}

The flux equations for the model are based on \ref{eq:thermokinetic_flux_forward}, meaning substrate saturation was assumed. This leaves the katalytic constants, enzyme concetration and thermodynamics as the impacting variables.
For reactions the dG values were calculated with the the dG0 for the reactions, the gas constant, temperature and the product and substarte concentrations.
\begin{equation}
    dG == dG0 + RT * (Products - Substrate)
    \label{eq:general_dG}
\end{equation}

With the gas constant, temperature and dG0 being constant values the change in the thermodynamic term comes from changes in substrate and product concentrations. 

The objective to maximize in the model was the glutamine synthesis as a stand in for the growth rate of the cell. Glutamine was chosen because it is a precursor to many other amino acids as well as a substrate for other metabolic reactions that are important for cell growth.
It was assumed that most reactions can only go in a forward direction to simplify the model. 
The protome of the system is limited by a density constraint simulating the limited space of a cell. With that no singular enzyme concentration could not exceed a set upper boundary, while also contraining the complete proteome with this limit.
For the baseline model no intracellular concentrations, fluxes or dG of reaction were fixed to a specific value. The metabolite concentrations are fixed between log(1e-9) and log(100e-3) (log(M)), the fluxes between -1 mmol/gDW/h and 100 mmol/gDW/h (gDW = gramm dry weight) and dG of reaction between -80 kJ/mol and 1 kJ/mol. 

The base model shows that under a changing ratio of ATP to ADP the proteome stays overall consinstent, with only slight increases and decreases in a few enzymes (see \ref{fig:proteome_upper_atp} and \ref{fig:proteome_lower_atp}). The metabolite concentations change more drametically due to the ratio change, which are shown in \ref{fig:metabolites_upper_atp} and \ref{fig:metabolites_lower_atp}. 
\begin{figure}[H]
    \centering
    \includegraphics[width=\textwidth]{imgs/EMP-model/Upper_Glycolysis_Proteome_ATP_ADP.pdf}
    \caption{Concentrations of the enzymes in the upper glycolysis under the change in ATP/ADP ratio from 0.1 M to 10.0 M.}
    \label{fig:proteome_upper_atp}
\end{figure}

\begin{figure}[H]
    \centering
    \includegraphics[width=\textwidth]{imgs/EMP-model/Lower_Glycolysis_Proteome_ATP_ADP.pdf}
    \caption{Concentrations of the enzymes in the lower glycolysis under the change in ATP/ADP ratio from 0.1 M to 10.0 M.}
    \label{fig:proteome_lower_atp}
\end{figure}

\begin{figure}[H]
    \centering
    \includegraphics[width=\textwidth]{imgs/EMP-model/Upper_Glycolysis_Metabolites_ATP_ADP.pdf}
    \caption{Metabolite concentrations of the upper glycolysis over the change of ATP/ADP ratio from 0.1 M to 10.0 M.}
    \label{fig:metabolites_upper_atp}
\end{figure}

\begin{figure}[H]
    \centering
    \includegraphics[width=\textwidth]{imgs/EMP-model/Lower_Glycolysis_Metabolites_ATP_ADP.pdf}
    \caption{Metabolite concentrations of the lower glycolysis over the change of ATP/ADP ratio from 0.1 M to 10.0 M.}
    \label{fig:metabolites_lower_atp}
\end{figure}

Similarly the metabolite concentrations for the NADH/NAD ratio change in a comparabble way to the ones seen for the ATP/ADP ratio. The change seen for the upper glycolysis (\ref{fig:metabolites_upper_nadh}) is less rapid and the concentrations of g6p and f6p do not drop as low as seen in \ref{metabolites_upper_atp}. For the lower glycolysis the metabolite concentration change (\ref{fig:metabolites_lower_nadh}) is smoother for the NADH/NAD ratio when compared to that of the ATP/ADP ratio (\ref{fig:metabolites_lower_atp}), but the genreal behavior is similar.
\begin{figure}[H]
    \centering
    \includegraphics[width=\textwidth]{imgs/EMP-model/Upper_Glycolysis_Metabolites_NADH_NAD.pdf}
    \caption{Metabolite concentrations of the upper glycolysis over the change of NADH/NAD ratio from 0.05 M to 10.0 M.}
    \label{fig:metabolites_upper_nadh}
\end{figure}

\begin{figure}[H]
    \centering
    \includegraphics[width=\textwidth]{imgs/EMP-model/Lower_Glycolysis_Metabolites_NADH_NAD.pdf}
    \caption{Metabolite concentrations of the lower glycolysis over the change of NADH/NAD ratio from 0.05 M to 10.0 M.}
    \label{fig:metabolites_lower_nadh}
\end{figure}

As it was seen for the proteome of the ATP/ADP ratio the proteome stays relatively constant over all different values for the NADH/NAD ratio. The relative concentrations are for both the upper and lower glycolysis are even comparable to the ones seen in \ref{fig:proteome_upper_atp} and \ref{fig:proteome_lower_atp}.
\begin{figure}[H]
    \centering
    \includegraphics[width=\textwidth]{imgs/EMP-model/Upper_Glycolysis_Proteome_NADH_NAD.pdf}
    \caption{Concentrations of the enzymes in the upper glycolysis under the change in ATP/ADP ratio from 0.1 M to 10.0 M.}
    \label{fig:proteome_upper_nadh}
\end{figure}

\begin{figure}[H]
    \centering
    \includegraphics[width=\textwidth]{imgs/EMP-model/Lower_Glycolysis_Proteome_NADH_NAD.pdf}
    \caption{Concentrations of the enzymes in the lower glycolysis under the change in ATP/ADP ratio from 0.1 M to 10.0 M.}
    \label{fig:proteome_lower_nadh}
\end{figure}

The relatively constant proteome can be explained by the big changes in metabolite concentrations. With these concentration having a direct effect on the termodynamic term of the flux calculation, the fluxes still change but only due to the thermodynamic term. Thus the fluxes will be adjusted to the new conditions without the need for the system to produce or destroy any enzymes. In a cell this would also make sense, due to the fact that the synthesis of enzymes is expensive, while using or importing/exporting more metabolites to change the fluxes of reactions has a similar effect with less cost to the cell.
This conection can be seen in \ref{fig:upper_fluxes_atp} which shows that the fluxes for the ractions stay the same even under this change in ATP/ADP ratio.

\begin{figure}[H]
    \centering
    \includegraphics[width=\textwidth]{imgs/EMP-model/Upper_Glycolysis_Fluxes_ATP_ADP.pdf}
    \caption{Plot of the change in fluxes of the upper glycolysis reactions due to the change in ATP/ADP ratio.}
    \label{fig:upper_fluxes_atp}
\end{figure}

This effect can not only be seen when looking at a change in the ATP/ADP ratio but also for a change in the ratio of NADH/NAD (\ref{fig:upper_fluxs_nadh}).
\begin{figure}[H]
    \centering
    \includegraphics[width=\textwidth]{imgs/EMP-model/Upper_Glycolysis_Fluxes_ATP_ADP.pdf}
    \caption{Plot of the change in fluxes of the upper glycolysis reactions due to the change in NADH/NAD ratio from 0.05 M to 10.0 M.}
    \label{fig:upper_fluxes_nadh}
\end{figure}

These behaiviors are not only observable for the fluxes and concentrations of the glycolysis reactions. They can also be seen if we compare the proteome and fluxes between the ratios for the glutamine synthesis, which means this behaivior is not exclusive to the gycolosis but fundamental to the system.
In some of these plots not all concentrations are visible, but this is explainable with the fact that those concentrations of enzymes or metabolites or the fluxes are so low that they are comperativly small and don't show up in a graph that only shows relative concentrations.

The trend of a mostly stable relative proteome continues even when the cofactors are fixed, while the matabolite concentrations show a similar behavior to the system without fixed cofactor concentrations. This also has the same effect on the relative fuxes, meaning they do not change as a result. The only thing that shows a difference to the base model are the relative cofactor concentrations themselves, which makes sense, if they are more limited in their range than before that would mean that the ratio between them change. They also show similar changes in cofactor concentrations between themselves, with the plots for fixed ATP (\ref{fig:cofactor_atp_fixed_atp}) and ADP (\ref{fig:cofactor_adp_fixed_atp}) being almost the same, with the slight difference in the NADH concentration maximum. 
The values to which the cofactors are fixed were taken from \cite{bennett2009absolute}. With these fixed ATP and ADP concentrations the ratio change for NADH/NAD also show generally the same trend, with the ATP concentration being lower when the ATP concentration is fixed. The relative NADH and NAD concentrations resembeling those of the system with fixed ATP and ADP.
\begin{figure}[H]
    \centering
    \includegraphics[width=\textwidth]{imgs/EMP-model/ATP_fixed/Cofactor_Metabolites_ATP_ADP.pdf}
    \caption{Change of the Cofactor concentration with a change in ATP/ADP ratio (0.1 M to 10.0 M.) and ADP fixed between 1/4.0 * 9.6e-3 M and 4.0 * 9.6e-3 M.}
    \label{fig:cofactor_atp_fixed_atp}
\end{figure}

\begin{figure}[H]
    \centering
    \includegraphics[width=\textwidth]{imgs/EMP-model/ADP_fixed/Cofactor_Metabolites_ATP_ADP.pdf}
    \caption{Change of the Cofactor concentration with a change in ATP/ADP ratio (0.1 M to 10.0 M.) and ADP fixed to 4.0 * 5.6e-4 M.}
    \label{fig:cofactor_adp_fixed_atp}
\end{figure}

Comperable to the the system with fixed ATP and ADP, similar trends are seen in when NADH or NAD are fixed. For this NADH was fixed to 4.0 * 8.3e-5 M and the NAD concentration between 1/4.0 * 2.6e-3 M and 4.0 * 2.6e-3 M. Here the general trends are similar, but unlike in the system with fixed ATP and ADP the relative concentration differ between the fixed NADH and NAD. Specifically the relatvie ATP and ADP concentration are distictly higher when the NADH concentration is fixed. This can be explained by the fact that the fixed NADH concentation is about 1e-2 M lower than the fixed NAD concentartion.

Further the effect of fixing the internal metabolite concentrations under the change in ATP/ADP ratio was investegated, as well as the NADH/NAD ratio. As with the base model and the fixed cofactor models, fixing metabolites to specific values shows little effect on the proteome and fluxes. The metabolite and cofactor concentrations again show combarable trends between systems. Looking at the examle of setting fructose-1,6-bishosphate to a fixed range of 1/4.0 * 1.5e-2 M and 4.0 * 1.5e-2 M, we find that the relative proteome fractions change in the same way as the base model. Shown are examples for this behavior for the upper glycolysis enzymes in \ref{fig:upper_proteome_fdp_fixed_atp} and \ref{fig:upper_proteome_fdp_fixed_nadh}.
\begin{figure}[H]
    \centering
    \includegraphics[width=\textwidth]{imgs/EMP-model/FDP_fixed/Upper_Glycolysis_Proteome_ATP_ADP.pdf}
    \caption{Relative proteome concentration of the upper glycolysis enzymes under ATP/ADP ratio change and fixed fructose-1,6-bisphosphate.}
    \label{fig:upper_proteome_fdp_fixed_atp}
\end{figure}

\begin{figure}[H]
    \centering
    \includegraphics[width=\textwidth]{imgs/EMP-model/FDP_fixed/Upper_Glycolysis_Proteome_NADH_NAD.pdf}
    \caption{Relative proteome concentration of the upper glycolysis enzymes under NADH/NAD ratio change and fixed fructose-1,6-bisphosphate.}
    \label{fig:upper_proteome_fdp_fixed_nadh}
\end{figure}

Fixing other metabolite concentrations for example that of phosphoenolpyruvate leads to very similar results in relative proteome concentrations. Complementary the metabolite concentartion changes are mostly in line with the chages seen in the base model when the ATP/ADP and NADH/NAD ratio are varied. These relative concentration changes do show differences between the different fixed metabolite concentrations. The general trends are comparablewhen looking at the ATP/ADP ratio (\ref{fig:upper_metabolites_fdp_fixed_atp}), but for a change in NADH/NAD ratio leads to bigger differences (\ref{fig:upper_metabolites_fdp_fixed_nadh}). 
\begin{figure}[H]
    \centering
    \includegraphics[width=\textwidth]{imgs/EMP-model/FDP_fixed/Upper_Glycolysis_Metabolites_ATP_ADP.pdf}
    \caption{Relative metabolite concentration in the upper glycolysis under ATP/ADP ratio change and fixed fructose-1,6-bisphosphate.}
    \label{fig:upper_metabolites_fdp_fixed_atp}
\end{figure}

\begin{figure}[H]
    \centering
    \includegraphics[width=\textwidth]{imgs/EMP-model/FDP_fixed/Upper_Glycolysis_Metabolites_NADH_NAD.pdf}
    \caption{Relative metabolite concentration in the upper glycolysis under NADH/NAD ratio change and fixed fructose-1,6-bisphosphate.}
    \label{fig:upper_metabolites_fdp_fixed_nadh}
\end{figure}

Bigger changes can be obsereved with phosphoenolpyruvate (PEP) being fixed to a range between 1/4.0 * 1.8e-4 M and 4.0 * 1.8e-4 M. Here the relative proteome as well as the relative metabolite concentrations show changes. With the most promenent changes happening to the relative proteome of the upper (\ref{fig:upper_proteome_pep_fixed_atp}) and lower glycolysis (\ref{fig:lower_proteome_pep_fixed_atp}) as well as the gultamine synthesis (\ref{fig:glutamine_proteome_pep_fixed_atp}) under ATP/ADP ratio changes as well as the realative metabolite concentrations to the lower glycolysis metabolites under both ratios (\ref{fig:lower_metabolites_pep_fixed_atp}, \ref{fig:lower_metabolites_pep_fixed_nadh}) and the upper glycolysis with NADH/NAD ratio change (\ref{fig:upper_metabolite_pep_fixed_nadh}). 
\begin{figure}[H]
    \centering
    \includegraphics[width=\textwidth]{imgs/EMP-model/PEP_fixed/Upper_Glycolysis_Proteome_ATP_ADP.pdf}
    \caption{Relative proteome of the upper glycolysis with a fixed PEP concentration and a variying ATP/ADP ratio.}
    \label{fig:upper_proteome_pep_fixed_atp}
\end{figure}

\begin{figure}[H]
    \centering
    \includegraphics[width=\textwidth]{imgs/EMP-model/PEP_fixed/Glutamine_synthesis_Proteome_ATP_ADP.pdf}
    \caption{Relative proteome for the glutamine synthesis with a fixed PEP concentration and a variying ATP/ADP ratio.}
    \label{fig:glutamine_proteome_pep_fixed_atp}
\end{figure}

\begin{figure}[H]
    \centering
    \includegraphics[width=\textwidth]{imgs/EMP-model/PEP_fixed/Lower_Glycolysis_Proteome_ATP_ADP.pdf}
    \caption{Relative proteome of the lower glycolysis with a fixed PEP concentration and a variying ATP/ADP ratio.}
    \label{fig:lower_proteome_pep_fixed_atp}
\end{figure}

\begin{figure}[H]
    \centering
    \includegraphics[width=\textwidth]{imgs/EMP-model/PEP_fixed/Lower_Glycolysis_Metabolites_ATP_ADP.pdf}
    \caption{Relative metabolite concentrations of the lower glycolysis with a fixed PEP concentration and a variying ATP/ADPratio.}
    \label{fig:lower_metabolites_pep_fixed_atp}
\end{figure}

\begin{figure}[H]
    \centering
    \includegraphics[width=\textwidth]{imgs/EMP-model/PEP_fixed/lower_Glycolysis_Metabolites_NADH_NAD.pdf}
    \caption{Relative metabolite concentrations of the lower glycolysis with a fixed PEP concentration and a variying NADH/NAD ratio.}
    \label{fig:lower_metabolites_pep_fixed_nadh}
\end{figure}

\begin{figure}[H]
    \centering
    \includegraphics[width=\textwidth]{imgs/EMP-model/PEP_fixed/Upper_Glycolysis_Metabolites_NADH_NAD.pdf}
    \caption{Relative metabolite concentrations of the upper glycolysis with a fixed PEP concentration and a variying NADH/NAD ratio.}
    \label{fig:upper_metabolite_pep_fixed_nadh}
\end{figure}

These models show the same relative flux distributions as the base model, and those with fixed cofactor concentrations. This supports the assumption that the fluxes are kept stable due to the change in metabolite concentrations, while the relative enzyme concentrations stay the same.

\section{Results}
The results for the different models show that the protome is consistent over all different metabolite concentations and cofactor ratios. This leads to the fact that the concentrations of the intracellular metabolites are adjusted by the system to keep the relative reaction fluxes at stable rates. 
This behaivior can be explained by assuming that it is less cost efficient for the cell to synthesize or desythesize enzymes and it is better to use up different amounts of the metabolites. 

